\chapter{Программный код тестов на языке Java}\label{appendix-extra-examples}

Код юнит тестов.

\begin{lstlisting}[language=Java]
package io.jenkins.plugins.sample;

import hudson.model.*;
import hudson.tasks.Shell;
import hudson.util.RunList;
import org.junit.Rule;
import org.junit.Test;
import org.jvnet.hudson.test.JenkinsRule;

import java.text.DateFormat;
import java.text.ParseException;
import java.text.SimpleDateFormat;
import java.util.*;

public class BuildConfigurationStatisticsBuilderTest {

    @Rule
    public JenkinsRule jenkins = new JenkinsRule();

    @Test
    public void testWorkingSystem() {
        assert 1 == 1;
    }

    @Test
    public void testSuccessBuildFromCustomBuild() throws Exception {
        FreeStyleProject project = jenkins.createFreeStyleProject();
        project.getBuildersList().add(new BuildConfigurationStatisticsBuilder());
        jenkins.buildAndAssertSuccess(project);
    }

    @Test
    public void testFailBuildFromCustomBuild() throws Exception {
        FreeStyleProject project = jenkins.createFreeStyleProject();
        project.getBuildersList().add(new Shell("echo1 hello"));
        jenkins.buildAndAssertStatus(Result.FAILURE, project);
    }

    @Test
    public void testConvertLongTimeToDate() throws ParseException {
        DateFormat dateFormat = new SimpleDateFormat("dd/MM/yyyy");
        Date date = dateFormat.parse("23/09/2007");
        long time = date.getTime();
        long resultDate = DateTimeHandler.convertDateToLongTime(date);
        assert resultDate == time;
    }

    @Test
    public void testConvertDateToLongTime() throws ParseException {
        DateFormat dateFormat = new SimpleDateFormat("dd/MM/yyyy");
        Date date = dateFormat.parse("23/09/2007");
        long time = date.getTime();
        Date resultDate = DateTimeHandler.convertLongTimeToDate(time);
        assert resultDate.equals(date);
    }
    @Test
    public void testGetDayOfMonth() throws ParseException {
        DateFormat dateFormat = new SimpleDateFormat("dd/MM/yyyy");
        Date date = dateFormat.parse("23/09/2007");
        Date date2 = dateFormat.parse("29/02/2008");
        int daysDate = DateTimeHandler.getDayOfMonth(date);
        int daysDate2 = DateTimeHandler.getDayOfMonth(date2);
        assert daysDate == 23;
        assert daysDate2 == 29;
    }

    @Test
    public void testGetCurrentMonthDays()  {
        int daysDate = DateTimeHandler.getCurrentMonthDays();
        Calendar mycal = new GregorianCalendar();
        int daysInMonth = mycal.getActualMaximum(Calendar.DAY_OF_MONTH);
        assert daysDate == daysInMonth;
    }

    @Test
    public void testGetLastMonthDays(){
        int daysDate = DateTimeHandler.getLastMonthDays();
        Date now = new Date();
        Calendar c = Calendar.getInstance();
        c.setTime(now);
        c.add(Calendar.MONTH, -1);
        int daysInMonth = c.getActualMaximum(Calendar.DAY_OF_MONTH);
        assert daysDate == daysInMonth;
    }

    @Test
    public void testDateToString() throws ParseException {
        DateFormat dateFormat = new SimpleDateFormat("dd/MM/yyyy");
        Date date = dateFormat.parse("23/09/2007");
        String strDate = DateTimeHandler.dateToString(date, "dd-MM-yyyy");
        assert strDate.equals("23-09-2007");
    }

    @Test
    public void testDateMonthToString() throws ParseException {
        DateFormat dateFormat = new SimpleDateFormat("dd/MM/yyyy");
        Date date = dateFormat.parse("23/09/2007");
        String strDate = DateTimeHandler.dateMonthToString(date);
        assert strDate.equals("2007-09");
    }

    @Test
    public void testCreateDateMonthMap()  {
        int daysDate = DateTimeHandler.getLastMonthDays();
        HashMap<String, List<Double>> dictDateMonthZero = DateTimeHandler.createDateMonthMap();
        assert  dictDateMonthZero.size() == daysDate;
        assert  !dictDateMonthZero.isEmpty();
        for (Map.Entry<String, List<Double>> entry : dictDateMonthZero.entrySet()) {
            assert entry.getValue().isEmpty();
        }
    }

    @Test
    public void testCreateDateWeekMapSuccessRate()  {

        HashMap<String, HashMap<String, Integer>> dictDateMonthZero = DateTimeHandler.createDateWeekMapSuccessRate();
        assert  dictDateMonthZero.size() == 7;
        assert  !dictDateMonthZero.isEmpty();
        for (Map.Entry<String, HashMap<String, Integer>> entry : dictDateMonthZero.entrySet()) {
            assert entry.getValue().equals(new HashMap(){{
                put("fail", 0);
                put("success", 0);
            }});
        }
    }

    @Test
    public void testCreateDateMonthMapSuccessRate()  {
        int daysDate = DateTimeHandler.getLastMonthDays();
        HashMap<String, HashMap<String, Integer>> dictDateMonthZero = DateTimeHandler.createDateMonthMapSuccessRate();
        assert  dictDateMonthZero.size() == daysDate;
        assert  !dictDateMonthZero.isEmpty();
        for (Map.Entry<String, HashMap<String, Integer>> entry : dictDateMonthZero.entrySet()) {
            assert entry.getValue().equals(new HashMap(){{
                put("fail", 0);
                put("success", 0);
            }});
        }
    }

    @Test
    public void testCreateDateMonthMapTestCount()  {
        int daysDate = DateTimeHandler.getLastMonthDays();
        HashMap<String, Integer> dictDateMonthZero = DateTimeHandler.createDateMonthMapTestCount();
        assert  dictDateMonthZero.size() == daysDate;
        assert  !dictDateMonthZero.isEmpty();
        for (Map.Entry<String, Integer> entry : dictDateMonthZero.entrySet()) {
            assert entry.getValue() == 0;
        }
    }

    @Test
    public void testGetTimeInQueue() throws Exception {
        FreeStyleProject project = jenkins.createFreeStyleProject();
        project.getBuildersList().add(new BuildConfigurationStatisticsBuilder());
        jenkins.buildAndAssertSuccess(project);
        Run run = project.getBuilds().getLastBuild();
        long time = new TimeInQueueFetcher().getTimeInQueue(run);
        long queuedTime = run.getStartTimeInMillis() - run.getTimeInMillis();
        assert time == queuedTime;
    }

    @Test
    public void testCreateDateYearMap()  {
        HashMap<String, List<Double>> dictDateYearZero = DateTimeHandler.createDateYearMap();
        assert  dictDateYearZero.size() == 12;
        assert  !dictDateYearZero.isEmpty();
        for (Map.Entry<String, List<Double>> entry : dictDateYearZero.entrySet()) {
            assert entry.getValue().isEmpty();
        }
    }

    @Test
    public void testCreateDateWeekMap()  {
        HashMap<String, List<Double>> dictDateWeekZero = DateTimeHandler.createDateWeekMap();
        assert  dictDateWeekZero.size() == 7;
        assert  !dictDateWeekZero.isEmpty();
        for (Map.Entry<String, List<Double>> entry : dictDateWeekZero.entrySet()) {
            assert entry.getValue() .isEmpty();
        }
    }

    @Test
    public void testCreateDateYearMapSuccessRate()  {
        HashMap<String, HashMap<String, Integer>> dictDateYearZero = DateTimeHandler.createDateYearMapSuccessRate();
        assert  dictDateYearZero.size() == 12;
        assert  !dictDateYearZero.isEmpty();
        for (Map.Entry<String, HashMap<String, Integer>> entry : dictDateYearZero.entrySet()) {
            assert entry.getValue().equals(new HashMap(){{
                put("fail", 0);
                put("success", 0);
            }});
        }
    }

    @Test
    public void testFilterPeriodBuild() throws Exception {
        FreeStyleProject project = jenkins.createFreeStyleProject();
        project.getBuildersList().add(new BuildConfigurationStatisticsBuilder());
        jenkins.buildAndAssertSuccess(project);
        jenkins.buildAndAssertSuccess(project);
        List<Run> runList = new RunList<>(project);

        BuildLogic instance1 = new BuildLogic(IntervalDate.WEEK, true, (RunList<Run>) runList);
        instance1.filterPeriodBuild();

        assert  instance1.buildList.size() == 2;

        BuildLogic instance2 = new BuildLogic(IntervalDate.ALL, true, (RunList<Run>) runList);
        instance2.filterPeriodBuild();

        assert  instance2.buildList.size() == 2;

        BuildLogic instance3 = new BuildLogic(IntervalDate.MONTH, true, (RunList<Run>) runList);
        instance3.filterPeriodBuild();

        assert  instance3.buildList.size() == 2;

        BuildLogic instance4 = new BuildLogic(IntervalDate.YEAR, true, (RunList<Run>) runList);
        instance4.filterPeriodBuild();

        assert  instance4.buildList.size() == 2;
    }
    @Test
    public void testFilterFailedBuild() throws Exception {
        FreeStyleProject project = jenkins.createFreeStyleProject();
        project.getBuildersList().add(new BuildConfigurationStatisticsBuilder());
        jenkins.buildAndAssertSuccess(project);
        project.getBuildersList().add(new Shell("echo1 hello"));
        jenkins.buildAndAssertStatus(Result.FAILURE, project);
        List<Run> runList = new RunList<>(project);
        for (Run run :runList) {
            System.out.println(run.getResult());
        }

        BuildLogic instance1 = new BuildLogic(IntervalDate.WEEK, false, (RunList<Run>) runList);
        instance1.filterFailedBuild();

        assert  instance1.buildList.size() == 1;

    }

    @Test
    public void testCreateDateQuarterMap()  {
        HashMap<String, List<Double>> dictDateQuarterZero = DateTimeHandler.createDateQuarterMap();
        assert  dictDateQuarterZero.size() == 4;
        assert  !dictDateQuarterZero.isEmpty();
        for (Map.Entry<String, List<Double>> entry : dictDateQuarterZero.entrySet()) {
            assert entry.getValue().isEmpty();
        }
    }

    @Test
    public void testCreateDateDayMap() throws ParseException {
        HashMap<String, List<Double>> dictDateDayZero = DateTimeHandler.createDateDayMap();
        assert  dictDateDayZero.size() == 24;
        assert  !dictDateDayZero.isEmpty();
        for (Map.Entry<String, List<Double>> entry : dictDateDayZero.entrySet()) {
            System.out.println(entry.getKey());
            assert entry.getValue().isEmpty();
        }
    }
}


\end{lstlisting}

Код BDD тестов на Java.

\begin{lstlisting}[language=Java]

package io.jenkins.plugins.sample;
import io.cucumber.java.Before;
import io.cucumber.java.PendingException;
import io.cucumber.java.ru.*;
import com.google.gson.Gson;
import com.google.gson.reflect.TypeToken;
import hudson.model.*;
import hudson.tasks.Shell;
import hudson.util.RunList;
import org.apache.commons.lang.time.DateUtils;
import org.assertj.core.api.Assertions;
import org.junit.Test;
import org.junit.jupiter.api.extension.ExtendWith;
import org.junit.runner.RunWith;
import org.jvnet.hudson.test.JenkinsRule;
import org.mockito.Mock;
import org.mockito.Mockito;
import org.mockito.junit.MockitoJUnit;
import org.mockito.junit.MockitoJUnitRunner;
import org.mockito.junit.MockitoRule;

import static org.assertj.core.api.BDDAssertions.then;
import static org.mockito.ArgumentMatchers.any;
import static org.mockito.BDDMockito.given;
import static org.mockito.Mockito.mock;

import java.text.DateFormat;
import java.text.ParseException;
import java.text.SimpleDateFormat;
import java.util.*;


public class MyStepdefs {


    //@Rule public MockitoRule mockitoRule = MockitoJUnit.rule();

    Job job = mock(Job.class);

    FreeStyleBuild build = mock(FreeStyleBuild.class);
    FreeStyleBuild build2 = mock(FreeStyleBuild.class);
    FreeStyleBuild build3 = mock(FreeStyleBuild.class);
    FreeStyleBuild build4 = mock(FreeStyleBuild.class);


//    RunList<Run> buildList;

    private BuildDurationLogic buildDurationLogic;

    BuildConfigurationStatisticsAction buildConfigurationStatisticsAction;

    Date now;
    Date twoMonthAgo;

    Date fiveMonthAgo;
    Date twoWeekAgo;
    String formatNow;
    String formatTwoMonthAgo;
    String formatFiveMonthAgo;

    String formatNowQuarter;
    String formatTwoMonthAgoQuarter;
    String formatFiveMonthAgoQuarter;
    String formatTwoWeekAgo;

    Map<String, Double> map;
    Map<String, Object> jsonMap;


    @Before
    public void prepareData() throws ParseException {
        //подготовить данные
        now = new Date();
        twoMonthAgo = DateUtils.addMonths(now, -2);
        fiveMonthAgo = DateUtils.addMonths(now, -5);
        //twoWeekAgo = DateUtils.addWeeks(now, -2);

        formatNow = DateTimeHandler.dateToString(now, "yyyy-MM-dd");
        formatTwoMonthAgo = DateTimeHandler.dateToString(twoMonthAgo, "yyyy-MM-dd");
        formatFiveMonthAgo = DateTimeHandler.dateToString(fiveMonthAgo, "yyyy-MM-dd");
        //formatTwoWeekAgo = DateTimeHandler.dateToString(fiveMonthAgo, "yyyy-MM-dd");

        // quarter date
        formatNowQuarter = DateTimeHandler.dateToString(now, "yyyy-MM");
        formatTwoMonthAgoQuarter = DateTimeHandler.dateToString(twoMonthAgo, "yyyy-MM");
        formatFiveMonthAgoQuarter = DateTimeHandler.dateToString(fiveMonthAgo, "yyyy-MM");

        given(build.getStartTimeInMillis())
                .willReturn(DateTimeHandler.convertDateToLongTime(now));
        given(build2.getStartTimeInMillis())
                .willReturn(DateTimeHandler.convertDateToLongTime(twoMonthAgo));
        given(build3.getStartTimeInMillis())
                .willReturn(DateTimeHandler.convertDateToLongTime(fiveMonthAgo));
        given(build4.getStartTimeInMillis())
                .willReturn(DateTimeHandler.convertDateToLongTime(twoMonthAgo));

        given(build.getDuration())
                .willReturn(10000L);
        given(build2.getDuration())
                .willReturn(20000L);
        given(build3.getDuration())
                .willReturn(10000L);
        given(build4.getDuration())
                .willReturn(10000L);

        given(build.getResult())
                .willReturn(Result.FAILURE);

        given(build2.getResult())
                .willReturn(Result.SUCCESS);

        given(build3.getResult())
                .willReturn(Result.SUCCESS);

        given(build4.getResult())
                .willReturn(Result.SUCCESS);
    }

    @Дано("^выбраны параметры отображения за период \"([^\"]*)\" и с флагом отображения упавших сборок \"([^\"]*)\"$")
    public void получениеСборокЗаПериодИСФлагомОтображенияУпавшихСборок(IntervalDate period, Boolean failed) throws Throwable {

        RunList<Run> buildList = RunList.fromRuns(Arrays.asList(build, build2));
        given(job.getBuilds())
                .willReturn(buildList);
        buildDurationLogic = new BuildDurationLogic(period, failed, job.getBuilds());
    }

    @Когда("^выбран статистический показатель \"([^\"]*)\"$")
    public void выбранСтатистическийПоказатель(Statistics statistics) throws Throwable {
        map = buildDurationLogic.getBuildsDuration(statistics);
    }

    @Тогда("^отбираются успешные и упавшие сборки за месяц с вычислением суммарного времени$")
    public void отбираютсяУспешныеИУпавшиеСборкиЗаМесяцСВычислениемСуммарногоВремени() throws Throwable {
        then(map)
                .as("Check that map is not contain date two month ago entry with 20.0 time build duration")
                .doesNotContainEntry(formatTwoMonthAgo, 20.0)
                .as("Check that map is not contain date two month ago")
                .doesNotContainKey(formatTwoMonthAgo)
                .as("Check that map is contain date now")
                .containsKey(formatNow)
                .as("Check that map is contain date now value and initial values")
                .containsValues(10.0, 0.0)
                .as("Check that map is contain date now entry with 10.0 time build duration")
                .containsEntry(formatNow, 10.0)
                .as("Check that map has size how last month of days")
                .hasSize(DateTimeHandler.getLastMonthDays());
    }

    @Дано("^сформировано (\\d+) запуска задания$")
    public void сформированыЗапускиЗадания(int countRuns) throws Throwable {

        if (countRuns == 4) {
            RunList<Run> buildList2 = RunList.fromRuns(Arrays.asList(build, build2, build3, build4));
            given(job.getBuilds())
                    .willReturn(buildList2);
            buildConfigurationStatisticsAction = new BuildConfigurationStatisticsAction(job);
        } else {
            throw new PendingException();
        }
    }

    @Когда("^выбраны параметры отображения за период \"([^\"]*)\" и с флагом отображения упавших сборок \"([^\"]*)\" и статистический показатель \"([^\"]*)\"$")
    public void получениеСборокЗаПериодИСФлагомОтображенияУпавшихСборокПоПоказателю(String period, String failed, String statistics) throws Throwable {
        String jsonData = buildConfigurationStatisticsAction.getBuildDuration(period, failed, statistics);

        jsonMap = new Gson().fromJson(
                jsonData, new TypeToken<HashMap<String, Object>>() {}.getType()
        );
    }

    @Тогда("^отбираются успешные сборки за последние (\\d+) месяца с вычислением среднего времени$")
    public void отбираютсяУспешныеСборкиЗаКварталСВычислениемСреднегоВремени(int countMonth) throws Throwable {
        then(jsonMap)
                .as("Check that json is correct")
                .doesNotContainKey(formatFiveMonthAgoQuarter)
                .as("Check that map is not contain date two month ago")
                .doesNotContainEntry(formatNowQuarter, 10.0)
                .as("Check that map is contain date now")
                .containsKey(formatTwoMonthAgoQuarter)
                .as("Check that map is contain date now value and initial values")
                .containsValues(15.0)
                .as("Check that map is contain date now entry with 10.0 time build duration")
                .containsEntry(formatTwoMonthAgoQuarter, 15.0)
                .as("Check that map has size how last month of days")
                .hasSize(countMonth);
    }

}

\end{lstlisting}

Описание BDD тестов на Gherkin.

\begin{lstlisting}
# language: ru
Функция: Получение продолжительности выполнения сборок

  Сценарий: Получение продолжительности сборок за месяц
    Дано выбраны параметры отображения за период "MONTH" и с флагом отображения упавших сборок "true"
    Когда выбран статистический показатель "SUM"
    Тогда отбираются успешные и упавшие сборки за месяц с вычислением суммарного времени

  Сценарий: Получение среднего продолжительности сборок за квартал в JSON
    Дано сформировано 4 запуска задания
    Когда выбраны параметры отображения за период "QUARTER" и с флагом отображения упавших сборок "0" и статистический показатель "AVG"
    Тогда отбираются успешные сборки за последние 3 месяца с вычислением среднего времени
\end{lstlisting}


