\chapter*{Заключение} \label{ch-conclusion}
\addcontentsline{toc}{chapter}{Заключение}	% в оглавление 

В результате проведенной работы был разработан прототип плагина для визуализации статистики сборок Jenkins. Была проанализирована предметная область, проведен сравнительный анализ аналогичных решений.

Были выбраны средства и инструменты разработки, спроектирована архитектура плагина, описаны функциональные возможности, а также разработан программный код и интерфейс плагина.

В процессе проектирования и реализация были выбраны статистические показатели и типы диаграмм, по которым должна происходить визуализация метрик сборок Jenkins. Было проведено тестирование плагина различными методами и апробация на реальном проекте frontend-maven-plugin (более 4 тысяч звезд и 863 форка).

Плагин рекомендуется использовать для оценки эффективности написанного кода и тестирования, а также для оптимизации и улучшения процессов CI/CD, разработки и тестирования.

По итогу реализации объем кода проекта составляет $\sim$3220 строк, из которых:

\begin{itemize}
	\item $\sim$1680 строк  - Java код на сервере Jenkins;
	\item $\sim$780 строк - Jelly и JS на клиентской части;
	\item $\sim$260 строк - Java unit тесты;
	\item $\sim$210 строк - Java bdd тесты;
	\item $\sim$180 строк - Python код UI тестов;
	\item $\sim$85 строк - python скрипты для апробации.
\end{itemize}
