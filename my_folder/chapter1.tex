\chapter{Анализ средств сборки и визуализации программного обеспечения} \label{ch1}

% не рекомендуется использовать отдельную section <<введение>> после лета 2020 года
%\section{Введение. Сложносоставное название первого параграфа первой главы для~демонстрации переноса слов в содержании} \label{ch1:intro}

В первой главе рассмотрим:

\begin{enumerate}
	\item Процесс и инструменты сборки приложения.
	
	\item Обзор и сравнительный анализ средств CI/CD в контексте сборок приложения.
	
	\item Понятие статистик и визуализации сборок в контексте инструментов CI/CD.
	
	\item Обзор и сравнительный анализ плагинов Jenkins по визуализации статистики работы сборок.
	
	\item Требования к разработке.
	
	
\end{enumerate}

\section{Обзор и сравнительный анализ средств сборки программного обеспечения} \label{ch1:sec1}

Для осуществления сборки программного продукта существует множество инструментов, какое средство использовать определяют не только из преимуществ и недостатков этих средств, но и исходя из того, какой используется язык программирования, фреймворк и окружение.
На данный момент существует большое количество инструментов сборки приложения. 

Maven — инструмент для автоматизации сборки проектов, который используется с Java приложениями. Maven решает несколько проблем \cite{maven}:

\begin{itemize}
	\item  упрощение процесса сборки;
	\item обеспечение единой системы сборки;
	\item предоставление информации о проекте;
	\item упрощение работы с зависимостями, включая их автоматоическое обновление.
\end{itemize}

Gradle — система автоматизации сборки, которая также часто используется для Java разработки. Gradle включает в себя следуюзие возможности:

\begin{itemize}
	\item  упрощение процесса сборки;
	\item обеспечение единой системы сборки;
	\item предоставление информации о проекте;
	\item упрощение работы с зависимостями, включая их автоматоическое обновление.
\end{itemize}



\section{Особенности CI/CD систем} \label{ch1:sec2}

CI/CD — это технология автоматизации тестирования и доставки/развертывания готового приложения заказчику  \cite{cicd}. Данная технология стала неотъемлемой состовляющей DevOps методологии и помогает сократить временные  рескрсные затраты в процесе современного жизненного цикла приложения, когда до заказчика изначально доходит минимально жизнеспособный продукт (MVP) продукт, а затем дорабатывается с учетом новых требований заказчика, т.е. идет непрерывная разработка новых версий продукта.

Преимущества CI/CD подхода \cite{plusci}:

\begin{itemize}
	\item упрощение разработки - позволяет разработчикам распределять приоритеты и сконцентрироваться на самых важных аспектах;
	\item улучшение качества кода -  качество кодапроверяется до того, как он достигнет среды тестирования, проблемы в коде могут быть выявлены на ранних стадиях;
	\item более короткие циклы тестирования - меньший объем кода для проверки;
	\item более простой мониторинг изменений - меньший объем кода для проверки;
	\item более короткие циклы тестирования - становиться проще определить проблемы в процессе развертывания;
	\item более лекгий откат - меньшие усилия для отката приложения к предыдущей версии при возникновении проблем в новой версии.
\end{itemize}

Этапы разработки и принцип CI/CD подхода можно отразить с помощью рисунка 1.

\begin{figure}[ht!] 
	\center
	\includegraphics [scale=0.27] {my_folder/images//ci-cd}
	\caption{Цикл CI/CD} 
	\label{fig:ci-cd}  
\end{figure}

\section{Обзор и сравнительный анализ средств CI/CD} \label{ch1:sec3}



На данный момент существует множеств инстркментов CI/CD, которые обладают своими преимуществами и недостатками, были выделены самые распространенные системы:

\begin{itemize}
	\item TeamCity;
	\item Jenkins;
	\item GitLab CI;
	\item CircleCI;
	\item Bamboo.
\end{itemize}




Jenkins — это автономный сервер автоматизации с открытым исходным кодом, который можно использовать для автоматизации всех видов задач, связанных со сборкой, тестированием, доставкой или развертыванием программного обеспечения \cite{jenkins}.

TeamCity - это сервер CI от компании Jetbrains  \cite{tc}, который позволяет запускать параллельные сборки одновременно на разных платформах и средах, а также настраивать статистику по продолжительности сборки, уровню успешности, качеству кода и пользовательским метрикам.

GitLab CI - сервер CI от компании Gitlab \cite{gitlab}, которая также предоставляет одноименный репозиторий Git. GitLab CI/CD может обнаруживать ошибки на ранних этапах цикла разработки и гарантировать, что весь код, развернутый в рабочей среде, соответствует установленным стандартам кода.

CircleCI - сервер CI \cite{circle}, который позволяет настроить для эффективной работы очень сложных конвейеров кэширование, кэширование уровня Docker и классы ресурсов для работы на более быстрых машинах.

Bamboo — это инструмент непрерывной интеграции и доставки  \cite{bamboo}, который связывает автоматизированные сборки, тесты и выпуски в единый рабочий процесс.


Особое внимание следует уделить критерию OpenSource, этот критерий является достаточно важным с учетом, того, что многие компании после 2022 года ушли из РФ, тем самым стали либо недоступны, либо прекратили лицензировании и стали менее безопасными, т.к. новые версии продуктов больше недоступны и проблемы с безопасностью и другими дефектами не будут исправлены/доступны на территории РФ. Также критерий важен тем, что даже при наличии действия продуктов компаний, они обходилось крупным ИТ-компаниям достаточно дорого.

\begin{table}
    \centering
    \begin{tabular}{|p{3cm}|p{2cm}|p{2cm}|p{2cm}|p{2cm}|p{2cm}|}
    \hline
        Критерий & Jenkins & TeamCity & GitLab CI & CircleCI & Bamboo \\ \hline
        Открытый исходный код & + & - & - & - & - \\ \hline
        Цена & Бесплатно & от 45\$ в месяц & от 5\$ в месяц & от 15\$ в месяц & от 1200\$ в год \\ \hline
        Встроенный функционал & 2/5 & 5/5 & 4/5 & 4/5 & 4/5 \\ \hline
        Интеграции & 5/5 & 4/5 & 4/5 & 3/5 & 3/5  \\ \hline
        Поддержка репозиториев Git & Любой репозиторий & GitHub, GitLab, Bitbucket & GitHub, GitLab, Bitbucket  & GitHub, GitLab, Bitbucket & Любой репозиторий  \\ \hline

    \end{tabular}
\end{table}	

Критерий интеграция показывает сколько можно подключить к системе плагинов и интеграций со сторонними сервисами, а встроенный функционал сколько функций из коробки поддерживает, то или иное средство, наличие инструментов, которые позволят облегчить работу.

Среди всех средств особо ярко выделяется Jenkins, поскольку он является бесплатным и с открытым исходным кодом, а также обладает большим количество интеграций и плагинов, которые постоянно пишутся, что позволяет устранить основной его недостаток по наличию встроенных функций. После 2022 года в РФ это стал самый востребованный инструмент для настройки CI конвейеров, и обосновывает важность разработки плагина для устранения недостатков функциональности, которые есть в других средствах.
 
 
	
\section{Обзор и сравнительный анализ плагинов Jenkins по визуализации статистики сборок} \label{ch1:sec4}

В данной работе производится разрабтка плагина для визуализации метрик сборки Jenkins, основание для разработки является отсутсвие плагина, который полностью визуализирует метрики сборок в Jenkins, аналогично встроенному модулю в Teamcity. Многие российские ИТ-компании широко использовали TeamCity, этот модуль позволял отслеживать состояние отдельных конфигураций сборки с течением времени,собирать статистические данные по всей истории сборки и отображать их в виде наглядных диаграмм. В данной работе будет разработан плагин для воссоздания это модуля в Jenkins, с дополнительным функционалом, которого не хватало в TeamCity.

Сначала рассмотрим уже разработанные плагины визуализации и их недостатки и преимущества в сравнении с разрабатываемым решением.

\begin{table}
    \centering
    \begin{tabular}{|p{5cm}|p{2cm}|p{3cm}|p{3cm}|p{2cm}|}
    \hline
        Критерий & Build Monitor Plugin & Global Build Stats Plugin  & Build Time Blame & Плагин разрабатываемый  \\ \hline
        Наличие отслеживания аномальных результатов метрик  & - & - & - & +  \\ \hline
        Открытый исходный код  & + &+ & + & +  \\ \hline
        Визуализация времени выполнения и статуса последней сборки & + &+ & - (только время) &+  \\ \hline
        Визуализация Success Rate истории сборок & - & +/- (в Teamcity гистаграммы, которые показываею процентное соотношение нагляднее) & - & +  \\ \hline
       Визуализация Build Duration истории сборок (в числе average) & - & + & +  & +  \\ \hline
       Визуализация Time Spent in queue & - & - & -  &+  \\ \hline
      Визуализация Test Count & - & - & +  & +  \\ \hline
      Визуализация Artifacts Size & - &- &-  & +  \\ \hline
       Отображение всех графиков на одной странице по одному диапазону времени для наглядного отображения всех метрик в один момент и во времени & - & + & -  & +  \\ \hline


    \end{tabular}
\end{table}	

 Build Monitor Plugin - плагин, который обеспечивает наглядное представление статуса выбранных заданий Jenkins. Отображает состояние и ход выполнения выбранных заданий.
 
 Global Build Stats Plugin - плагин, который позволит собирать и отображать глобальную статистику результатов сборки, а также позволяющий отображать глобальную тенденцию сборки Jenkins/Hudson с течением времени.
 
  Build Time Blame - плагин, который сканирует вывод консоли на наличие успешных сборок и генерирует отчет, показывающий, как эти шаги повлияли на общее время сборки. Это предназначено для того, чтобы помочь проанализировать, какие этапы процесса сборки являются подходящими кандидатами на оптимизацию.
  
  После проведения сравнения аналогичных решений, были выявлены преимущества разрабатываемого плагина, которые обосновывают его разработку, это отсутсвие у данных плагинов функционала по визуализации Artifacts Size, Time Spent in queue, Success Rate истории сборок, а также наличие отслеживания аномальных результатов метрик, которое позволит определить проблемные сборки, у которых возникают временные проблемы с процессом CI/CD. Также данные плагины не предлагают динамическое изменение графиков по мере изменения временного интервала или установления фильтров.



\section{Требования к разработке} \label{ch1:sec5}

Поскольку разрабатываемый плагин является аналогом модуля статистики сборок в TeamCity, то функционал должен как минимум реализовывать функции модуля Statistics в TeamCity. В первую очередь должна производиться визуализация метрик сборок с помощью графиков и диаграм, а именно:

\begin{itemize}
	\item визуализация метрики success rate - процент успешности сборок, который будет показывать сколько сборок завершилось успешно;
	\item визуализациия метрики Build Duration - время выполнения сборок, в том числе должен быть доступен фильтр на добавления в график упавших сборок, а также возможность вычислять не только суммарно время сборок, а также среднее время всех сборок за определенный интервал времени;
	\item визуализациия метрики Time Spent in queue - время проведенное в очереди сборок, в том числе среднее время, вычисляемое аналогично Build Duration;
	\item  визуализациия метрики Test Count - количество выполненных тестов в сборке, в том числе количество выполненых тестов в упавших сборках, если таковые успели выполниться;
	\item визуализациия метрики Artifacts Size - размер созданных во время сборок артефактов, в том числе средний размер за определенный интервал времени, а также учет артефактов, которые умпели создаться в сборках до падения.
\end{itemize}

На всех графиках и диаграммах должна быть возможность выбора значения из выпадающего списка интервала времени, за который будет производиться сбор статистики за день, месяц, квартал, неделю, год и за весь промежуток времени, т.е. если, например, был выбран промежуток времени месяц, то должен выполняться следующий набор действий:

\begin{enumerate}
	\item Должна собираться информация о требуемой метрики у всех сборок.
	
	\item Производиться фильтрация сборок - т.е. отбираться только сборки за последний месяц (в том числе упавшие, если был выбран данный чекбокс).
	
	\item Полученные сборки должны групппироваться по дням т.е. на итоговом графике должно быть 30/31 точка или столбца.
	
	\item Если необходимо производиться вычисление среднего среди всех сгруппированных за день метрик сборок.
	
	\item Отображение всей информации о матриках сборки на одном графике или диаграмме.
	
	
\end{enumerate}

Также все графики должны располагаться друг под другом на однйо странице, что может наглядно показать (если на каждом графике был выбран один период), все вычесленные метрики за один период, например при выборе месяца все перечисленные метрики будут отображены на странице и можно будет увидеть, что происходило, например вчера по результатам запуска всех сборок.

Помимо реализации перечисленных функция, которые полностью аналогичны функциям TeamCity, плагин будет вычислять аномальные значения за определенный период - т.е. можно будет наглядно увидеть, например, в какие дни произошли сбои в работе сборок, это может быть например слишком большой размер артифактов у одной сборки за какой-то промежуток времени.

Также было принято решения добавить анализ данных, чтобы делать предположение, о том какими метриками будет обладать следующая запущенная сборка, при вычислении данного значения должно быть расчитаны веса каждой сборки/сборок по графику за определенный период, и если сборка была собрана, например, месяц назад - она должна иметь меньший вес, чем сборка, собранная вчера.

Также к разработке будет предъявлено требование об удобстве интерфейса: все графики должны быть удобными, не перегруженными информацией, а также интерфейс должен быть интуитивно понятно, чтобы данный плагин не усложнял восприятие собранной статистики сборок и не вызывал желание воспользоваться другим плагином или разработать другой более удобной, или отказаться от идеи смотреть статистику по сборкам.

\section{Выводы} \label{ch1:sec6}

По всем описанным выше разделам можно прийти к выводу, что данный плагин актуален для ИТ-компаний, которые ранее отдавали предпочтению многофункциональному инструменту TeamCity, в котором уже были все необходимые для работы функции, особенно это актуально для компания в РФ, но также может понадабиться и другим компания, которые приняли решение отказаться от TeamCity в пользу Jenkins из-за больших денежных затрат на лицензию. Также будет реализованы дополнительный функционал по сравнению с модулем TeamCity, что даст преимущества не только в цене. После проведенного обзора аналогичных решений становится понятно, что сейчас в Jenkins нет полнофункциональной замены модуля статистики TeamCit, также необходимо учесть и визуальную составляющую, чтобы при установке данного плагина разработчики выбирали его не только из-за отсутсвия другого решения.













%
%

%
%
%\begin{table} [htbp]% Пример оформления таблицы
%	\centering\small
%	\caption{Представление данных для сквозного примера по ВКР \cite{Peskov2004}}%
%	\label{tab:ToyCompare}		
%		\begin{tabular}{|l|l|l|l|l|l|}
%			\hline
%			$G$&$m_1$&$m_2$&$m_3$&$m_4$&$K$\\
%			\hline
%			$g_1$&0&1&1&0&1\\ \hline
%			$g_2$&1&2&0&1&1\\ \hline
%			$g_3$&0&1&0&1&1\\ \hline
%			$g_4$&1&2&1&0&2\\ \hline
%			$g_5$&1&1&0&1&2\\ \hline
%			$g_6$&1&1&1&2&2\\ \hline		
%		\end{tabular}	
%	\normalsize% возвращаем шрифт к нормальному
%\end{table}


% \firef{} от figure reference
% \taref{} от table reference
% \eqref{} от equation reference




