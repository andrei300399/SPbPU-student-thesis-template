\chapter*{Введение} % * не проставляет номер
\addcontentsline{toc}{chapter}{Введение} % вносим в содержание

Сегодня разработка информационных систем достаточно сложный процесс, который состоит из нескольких этапов: анализ требований заказчика, проектирование системы, разработка, тестирование и доставка приложения потенциальному заказчику. 

Для упрощение процесса работы в настоящий момент широко применяются практики DevOps одной из которых является CI/CD - непрерывная итеграция, сборка и доставка. Существует множество средств CI, которые применяются в промышленной разработке: TeamCity, Jenkins, Gitlab CI и другие.

Одним из лучших средств CI, в котором доступно много функций ''из коробки'' является TeamCity компании Jetbrains. TeamCity - мощный и сложный инструмент, который использовался крупными ИТ компаниями в промышленной разработке до 2022 года. Одним из главных недостатков TeamCity является то, что это платное решение, лицензия обходится ИТ компания достаточно дорого, также недостатком является то, что компания JetBrains покинула ИТ сектор РФ. Для того, чтобы преодолеть данные проблемы ИТ компании РФ начали поиск бесплатных средств с открытым исходным кодом. Одним из таких средств является Jenkins - средство CI, которое всегда пользовалось популярностью у разработчиков при локальной разработке решений с открытм исходным кодом. 

Jenkins обладает меньшим функционалом в сравнении с TeamCity, но имеет много подключаемых плагинов, которые могут помочь заменить или даже улучшить те функции, которые требуется разработчикам в процессе тестирования, сборки и доставки приложений.

\textbf{Актуальность исследования.} На данный момент в Jenkins нет плагина, который полностью заменяет модуль визуализации статистики Build Configuration Statistics. Часть плагинов реализует частичный функционал модуля TeamCity, подробнее о недостатках таких средств будет описано в сравнительном анализе и обзоре аналогов. Этот плагин/модуль требуется для того чтобы отслеживать состояние отдельных конфигураций сборки с течением времени, плагин собирает статистические данные по всей истории сборки и отображает их в виде наглядных диаграмм. 

В данной работе будет разработан плагин, который обеспечит визуализацию статистики работы сборок.

\textbf{Объект исследования} — сборки Jenkins с их метриками.

\textbf{Предмет исследования} — технологии разработки плагинов для Jenkins.

\textbf{Цель} - разработать плагин Jenkins для визуализации статистики сборок Jenkins.

Задачи:
 
\begin{enumerate}
	\item Изучить систему CI/CD Jenkins и работу и характеристики сборок Jenkins.
	
	\item Изучить методы разработки плагинов Jenkins.
	
	\item Провести проектировани плагина и описать архитектуру.
	
	\item Разработать код плагина.
	
	\item Провести тестирование и аппробацию плагина. 
	
\end{enumerate}




%% Вспомогательные команды - Additional commands
%\newpage % принудительное начало с новой страницы, использовать только в конце раздела
%\clearpage % осуществляется пакетом <<placeins>> в пределах секций
%\newpage\leavevmode\thispagestyle{empty}\newpage % 100 % начало новой строки