\chapter{Проектирование архитектуры плагина} \label{ch2}
	
% не рекомендуется использовать отдельную section <<введение>> после лета 2020 года
%\section{Введение} \label{ch2:intro}
В данной главе будет проведено проектирование разрабатываемого плагина: будет описана архитектура построения плагинов в Jenkins, а также архитектура разработки, будут выбраны инстркменты разработки, а также расмотрена функциональная модель системы. Поскольку плагин разрабатывается для системы Jenkins, то отладку и тестирование будем проводить в этой системе.

\section{Языки программирования} \label{ch1:sec1}

Для программирования плагина будет использоваться язык Java. Поскольку Jenkis написан на Java, то все плагины необходимо писать на том же языке. Это является главным минусом, а возможно и сложностью при разработке плагинов на Jenkins, поскольку ограничивает свободу разработчика.

Есть возможность разработки плагина с использованием языка программирования Groovy. Groov это динамический язык с возможностями статической типизации и статической компиляции для платформы Java\cite{groovy}, нацеленный на повышение производительности разработчиков, который плавно интегрируется с любой программой Java.

Недостатком такого выбора является то, что абсолютное большинство плагинов написано на чисто Java, а значит сообщества и поддержка при разработке на Java будет значительно большей. Также в сравнии с Groovy, Java обладает большей производительностью\cite{groovyvsjava}, статической типизацией и подходит для разработки приложений в парадигме ООП.

Java — это язык высокого уровня, который можно охарактеризовать следующими словами: объектно-ориентированный, многопоточный, динамический, высокпроизводительный и безопасный \cite{java}. Java используется для разработки высоконагруженный информационных систем, мобильных приложений, плагинов, десктопного ПО и др. К преимуществам Java также можно отнести компилируемость, что обеспечивает высокое быстродействие.

Java будет использоваться для программиования ядра плагина и бизнес-логики. Также для программирования графических компонентов, графиков и диаграмм будет использоваться язык программирования JavaScript. JS - это легковесный, интерпретируемый или JIT-компилируемый, объектно-ориентированный язык \cite{js}, основное предназначение которого выполнять сценарии на веб-страницах, что необходимо при разработке плагина, результаты которого отображаются на веб-страницах.

Помимо прочего, для стилизации компонентов веб-интерфейса будет использоваться язык каскадных таблиц стилей CSS \cite{css}, который позволит настроить удобное отображение и позиционирование элементов на странице плагина Jenkins. 

Верстка страниц будет осуществляться с помощью инструмента Jelly - все разрабатываемые плагины используют данный инструмент в Jenkins, поскольку с ним можно легко интегрировать Java, XML и JS. Jelly — это средство для преобразования XML в исполняемый код, это механизм сценариев и обработки на основе Java и XML \cite{jelly}. В Jelly можно вызывать функции Java, использовать такие синтакситеские конструкции, как циклы, условия и переменные, также он позволяет легко обратиться к объектам в Java.

\section{Инструменты сборки} \label{ch1:sec2}

В качетве инструмента сборки проекта был выбран Maven, который можно использовать для создания и управления любым проектом на основе Java. Преимущества Maven были описаны в первой главе при рассмотрении инструментов сборок приложения.

Абсолютное большинство разработанных плагинов для Jenkins использует Maven, поскольку Maven предоставляет удобные архетипы для начала разработки плагинов, что делает использование того же Gradle не рациональным.

\section{Библиотеки} \label{ch1:sec3}

Поскольку проект предполагает использовани графиков и диагармм, то необходимо было выбрать инструмент для работы с графиками в Jenkins и Java, который позволит отображать графики прямо на странице задания Jenkins. В качестве этого иструмента была выбрана библиотека Chart.js, которая на данный момент является самой популярной JavaScript библеотекой по оценкам GitHub и загрузок npm \cite{chartjs}. К преимуществам данной библиотеки можно отнести: 
\begin{itemize}
	\item у Chart.js очень подробная документация;
	\item отрисовка canvas делает Chart.js очень производительным, особенно для больших наборов данных и сложных визуализаций;
	\item строит отзывчивый интерфейс - перерисовывает диаграммы при изменении размера окна для идеальной детализации масштаба.
\end{itemize}

\section{Архитектура Jenkins} \label{ch1:sec4}

Перед объяснением построения архитектуры плагинов Jenkins, необходимо привести схему архитектуры Jenkins, где будет отображено место разрабатываемых плагинов в CI системе (рисунок 2.1). Установленные плагины Jenkins-CI, а также локальные сценарии и приложения выполняются на сервере Jenkins-CI и предоставляют расширяемый набор функций управления и обработки данных \cite{article}.

\begin{figure}[ht!] 
	\center
	\includegraphics [scale=0.47] {my_folder/images//ArchitectureJenkins}
	\caption{Архитектура Jenkins} 
	\label{fig:ArchitectureJenkins}  
\end{figure}


Архитектура плагинов использует точки расширения, которые, предоставляют разработчикам плагинов возможности реализации для расширения функциональности системы Jenkins \cite{atchplugin}. Точки расширения автоматически обнаруживаются Jenkins во время загрузки системы.

В разрабатываемомом плагине реализация будет происходить через класс Action. Actions являются основным строительным блоком расширяемости в Jenkins: их можно прикреплять ко многим объектам модели, хранить вместе с ними и при необходимости добавлять в их пользовательский интерфейс.

Помимо класса Action для того чтобы создать временные действия, которые будут прикреплены к заданию Jenkins будет использован класс TransientActionFactory, который позволяет создавать действия, которые будут отображаться на страницах Jenkins только при наличии соответствующего объекта - задания.

Разработка будет выполняться в объектно-ориентированной парадигме, т.е. приложение будет разбито на классы, будет применяться наследование, полиморфизм и инкапсуляция. Все классы, которые будут разработаны для плагина описаны в приложении П1.1. При рассмотрении диаграммы необходимо отметить, что два класса являются встроенными в Jenkins, это TransientActionFactoryб который позволяет добавлять действия к любому типу объекта, а также интерфейс Action - добавленный к объекту модели, создает дополнительное подпространство URL-адресов под родительским объектом модели, через которое он может взаимодействовать с пользователями. Actions также способны открывать доступ к левому меню в интерфейсы Jenkins, по которому обычно производится навигация при конфигурировании сборки. Для удобства использования плагина, предполагается добавить дополнительную ссылку в меню слева, для перехода на страницу визуализации метрик, а также динамически обновлять страницу при изменении параметров и фильтров, что и обосновывает использование данных встроенных классов.

Основная часть остальных классов требуется для работы с определенной метрикой статистики выполнения сборок Jenkins, что следует из их названия. Также будет разработан дополнительный класс DateTimeHandler, который позволит создать методы для удобной работы с датой и временем, что необходимо поскольку будет производиться преобразования одних типов дат к другим, сравнение дат между собой, а также получение определенных частей дат.

Функциональная модель в нотации Idef0 отображена в приложении П2.1.-3. Основное внимание на диаграмме уделяется визуализации статистики сборок, поскольку это изначально является целью разработки. Также там будут отражены дополнительные функции такие как фильтрация.

Диаграмма вариантов использования, показывающая функционал плагина отображена в приложении П3.1. На данной диаграмме основное внимание также уделяется процессу визуализации статистики метрик сборок. Основное действующее лицо одно - это пользователь системы, который запускает сборки и работает в CI системе, это может быть любой участник команды, который задействован в разработке, тестировании, доставке и внедрению приложения. В данном случае все эти роли представлены на диаграмме как разработчик.




%% Вспомогательные команды - Additional commands
%
%\newpage % принудительное начало с новой страницы, использовать только в конце раздела
%\clearpage % осуществляется пакетом <<placeins>> в пределах секций
%\newpage\leavevmode\thispagestyle{empty}\newpage % 100 % начало новой страницы